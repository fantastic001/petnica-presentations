\documentclass{beamer} 

\usepackage[utf8]{inputenc}
\usepackage{graphicx}
\usepackage{multicol}
\usepackage{listings}


\begin{document}

\title{Uvod u funkcionalno programiranje}
\author{Stefan Nožinić (stefan@lugons.org)}

\frame{
\titlepage
}

\frame{
\frametitle{Uvod}
\begin{itemize}
\item LISP 1950te
\item Deklarativni stil 
\item Imutabilnost 
\item Thread-safeness 
\end{itemize}
}

\begin{frame}[fragile]
\frametitle{Primer promene stanja u C}
\begin{lstlisting}
#include <stdio.h>

int f( int* x ) 
{
	int res = (*x) + 1;
	(*x)++;
	return res;
}

int main() {
	int x = 5;
	printf("%d\n", f(&x));
	printf("%d\n", f(&x));
	return 0;
}
\end{lstlisting}
\end{frame}

\frme{
\begin{itemize}
\item Funkcija f povećava x za jedan i vraća x+1
\item f menja stanje od x
\item Ako ne vidimo definiciju funkcije f, ne znamo šta ona zapravo radi sa vrednostima ulaznih parametara!
\item Da li uvek možemo da tvrdimo da je f(x) == f(x)?
\end{itemize}
}

\frame{
\frametitle{Prednosti nemenjanja stanja promenljivih}

\begin{itemize}
\item Kompajler će ispisati grešku kada pokušamo promeniti stanje
\item Svaka funkcija za iste ulazne parametre vraća istu povratnu vrednost
\item Moguće lako paralelizovati program jer nema promene stanja promenljive 
\end{itemize}
}

\frametitle{Funkcija koja vraća jedinicu ako je bar jedan ulaz jednak nuli}
\begin{lstlisting}
anyzero :: Int -> Int -> Int
anyzero 0 x = 1
anyzero x 0 = 1
anyzero x y = 0
\end{lstlisting}
\end{frame}


\begin{frame}[fragile]
\frametitle{Neke funkcije koje rade sa listama}
\begin{lstlisting}

mymap :: (a -> b) -> [a] -> [b]
mymap _ [] = []
mymap f (x:xs) = (f x) : (mymap f xs)


reduce :: (b -> a -> b) -> b -> [a] -> b 
reduce _ z [] = z 
reduce f z (x:xs) = reduce f (f z x) xs

myfilter :: (a -> Bool) -> [a] -> [a]
myfilter _ [] = [] 
myfilter f (x:xs) = if (f x) then x : (myfilter f xs) 
	else myfilter f xs

*Main> mymap (+ 1) [1,2,3]
[2,3,4]
*Main> filter (\x -> x `mod` 2 == 0) [1,2,3,4,5,6]
[2,4,6]
*Main> reduce (+) 0 [1,2,3,4,5]
15
*Main> 

\end{lstlisting}
\end{frame}

\end{document}
