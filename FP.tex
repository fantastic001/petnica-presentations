\documentclass{beamer} 

\usepackage[utf8]{inputenc}
\usepackage{graphicx}
\usepackage{multicol}
\usepackage{listings}


\begin{document}

\title{Uvod u funkcionalno programiranje}
\author{Stefan Nožinić (stefan@lugons.org)}

\frame{
\titlepage
}

\frame{
\frametitle{Uvod}
\begin{itemize}
\item LISP 1950te
\item Deklarativni stil 
\item Imutabilnost 
\item Thread-safeness 
\end{itemize}
}

\begin{frame}[fragile]
\frametitle{Primer promene stanja u C}
\begin{lstlisting}
#include <stdio.h>

int f( int* x ) 
{
	int res = (*x) + 1;
	(*x)++;
	return res;
}

int main() {
	int x = 5;
	printf("%d\n", f(&x));
	printf("%d\n", f(&x));
	return 0;
}
\end{lstlisting}
\end{frame}

\frame{
\begin{itemize}
\item Funkcija f povećava x za jedan i vraća x+1
\item f menja stanje od x
\item Ako ne vidimo definiciju funkcije f, ne znamo šta ona zapravo radi sa vrednostima ulaznih parametara!
\item Da li uvek možemo da tvrdimo da je f(x) == f(x)?
\end{itemize}
}

\frame{
\frametitle{Prednosti nemenjanja stanja promenljivih}

\begin{itemize}
\item Kompajler će ispisati grešku kada pokušamo promeniti stanje
\item Svaka funkcija za iste ulazne parametre vraća istu povratnu vrednost
\item Moguće lako paralelizovati program jer nema promene stanja promenljive 
\end{itemize}
}

\frame{
\frametitle{Data in - data out model} 
\begin{itemize}
\item Primenjivo nezavisno od jezika
\item Svaka funkcija vraća neku vrednost 
\item Za isti ulaz, funkcija uvek vraća isti izlaz
\item Funkcija ne zavisi od globalnog okruženja (globalne promenljive, IO, ...)
\item Funkcija ne modifikuje sopstveni ulaz 
\item Moguće primeniti i u OOP kodu - svaki setter vraća novi objekat 
\item Primer: lista
\end{itemize}
}

\begin{frame}[fragile]
\frametitle{Lista u C}
\begin{lstlisting}
typedef struct ListElem_t
{
        int value;
        struct ListElem_t* next;
} ListElem;
\end{lstlisting}
\end{frame}

\begin{frame}[fragile]
\frametitle{Konstruktor elementa liste}
\begin{lstlisting}

ListElem* create_element(int value, ListElem* next)
{
        ListElem* elem = (ListElem*) malloc(sizeof(ListElem));
        elem->value = value;
        elem->next = next;
        return elem;
}

\end{lstlisting}
\end{frame}


\begin{frame}[fragile]
\frametitle{Lista struktura}
\begin{lstlisting}

typedef struct
{
        ListElem* head;
} List;

\end{lstlisting}
\end{frame}

\begin{frame}[fragile]
\frametitle{Konstruktor liste}
\begin{lstlisting}
List* create_list(ListElem* head)
{
        List* l = (List*) malloc(sizeof(List));
        l->head = head;
        return l;
}
\end{lstlisting}
\end{frame}


\begin{frame}[fragile]
\frametitle{Dodavanje elementa u listu}
\begin{lstlisting}
List* push(List* l, int value)
{
        return create_list(create_element(
		value, l->head
	));
}
\end{lstlisting}
\end{frame}

\begin{frame}[fragile]
\frametitle{Rep liste}
\begin{lstlisting}
List* tail(List* l)
{
        return create_list(l->head->next);
}
\end{lstlisting}
\end{frame}


\begin{frame}[fragile]
\frametitle{print funkcija za ispis liste - zbog debug-a}
\begin{lstlisting}
void print(List* l)
{
        if (l->head == NULL) printf("\n");
        else
        {
                printf("%d ", l->head->value);
                print(tail(l));
        }
}
\end{lstlisting}
\end{frame}

\begin{frame}[fragile]
\frametitle{main}
\begin{lstlisting}
int main() {
        List *l = create_list(NULL);
        push(l, 5);
        print(l);
        List* l2 = push(l, 5);
        print(l2);

        print(tail(l2));
        print(push(l2, 10));
        return 0;
}
\end{lstlisting}
\end{frame}

\frame{
\begin{itemize}
\item Pure functional 
\item Funkcije ne modifikuju parametre 
\item Funkcije UVEK vraćaju rezultat
\item Funkcija može primiti funkciju kao parametar
\item Funkcija može vratiti funkciju kao rezultat
\item Kompajlira se ali se može i interpretirati interaktivno
\item ghc i ghci
\item ".hs" ekstenzija se obično koristi za fajlove sa kodom
\item Strogo tipovan - svaka vrednost mora imati svoj tip i jasno se zna šta je kog tipa tokom kompajliranja
\end{itemize}
}

\begin{frame}[fragile]
\frametitle{Primer funkcije koja računa kvadrat izlaza}
\begin{lstlisting}
sq :: Int -> Int
sq x = x*x
\end{lstlisting}
\end{frame}


\begin{frame}[fragile]
\frametitle{Primer upotrebe alata ghci}
\begin{lstlisting}
ghci example.hs 
>>> sq 5 
25
>>>
\end{lstlisting}
\end{frame}


\begin{frame}[fragile]
\frametitle{Primer funkcije koja računa zbir ulaznih parametara}
\begin{lstlisting}
add :: Int -> Int -> Int 
add a b = a+b
\end{lstlisting}
\end{frame}

\frame{
\begin{itemize}
\item Funkcija u Haskell-u prima samo jedan parametar
\item Funkcija `add` je funkcija koja prima jedan parametar i koja vraća funkciju koja prima drugi parametar i vraća zbir ta dva parametra
\item inc = add 1 je funkcija koja vraća ulaz uvećan za jedan
\end{itemize}
}

\frame{
\begin{itemize}
\item Jednostruko-povezane liste
\item l = [1,2,3]
\item head l 
\item tail l 
\item reverse l 
\item last l 
\item l = l1 ++ l2
\item 5 : l
\item range = [1..5]
\item range = [1, 1.01..5]
\end{itemize}
}


\begin{frame}[fragile]
\begin{lstlisting}
squares = [x*x | x <- [1..10]]

squaresDiv3 = [x*x | x <- [1..10], x*x `mod` 3 == 0]

unitcircle = [(x,y) | x <- [0, 0.01..10], y <- [0,0.01..10], x*x + y*y < 1]
\end{lstlisting}
\end{frame}

\frame{
\frametitle{Pattern matching}
\begin{itemize}
\item Fukciju je moguće definisati specifično za neke ulazne parametre
\item Neka vrsta zamene za "if" tako da kod izgleda lepo
\item Može se koristiti za "otpakivanje" podataka u složenim tipovima kao što je lista
\end{itemize}
}

\begin{frame}[fragile]
\frametitle{Faktorijel - primer pattern matching-a} 
\begin{lstlisting}
fact :: Int -> Int
fact 0 = 1
fact n = n * (fact (n-1))
\end{lstlisting}
\end{frame}

\begin{frame}[fragile]
\frametitle{Funkcija koja vraća jedinicu ako je bar jedan ulaz jednak nuli}
\begin{lstlisting}
anyzero :: Int -> Int -> Int
anyzero 0 x = 1
anyzero x 0 = 1
anyzero x y = 0
\end{lstlisting}
\end{frame}


\begin{frame}[fragile]
\frametitle{Neke funkcije koje rade sa listama}
\begin{lstlisting}

mymap :: (a -> b) -> [a] -> [b]
mymap _ [] = []
mymap f (x:xs) = (f x) : (mymap f xs)


reduce :: (b -> a -> b) -> b -> [a] -> b 
reduce _ z [] = z 
reduce f z (x:xs) = reduce f (f z x) xs

myfilter :: (a -> Bool) -> [a] -> [a]
myfilter _ [] = [] 
myfilter f (x:xs) = if (f x) then x : (myfilter f xs) 
	else myfilter f xs

*Main> mymap (+ 1) [1,2,3]
[2,3,4]
*Main> filter (\x -> x `mod` 2 == 0) [1,2,3,4,5,6]
[2,4,6]
*Main> reduce (+) 0 [1,2,3,4,5]
15
*Main> 

\end{lstlisting}
\end{frame}

\begin{frame}[fragile]
\frametitle{Funkcija koja vraća listu gde dva ista elementa nisu jedan do drugog} 
\begin{lstlisting}

norepeat :: (Eq a) => [a] -> [a]
norepeat [] = [] 
norepeat (x:[]) = [x]
norepeat (x:y:xs) = if x == y then norepeat (x:xs) else x:(norepeat (y:xs))

*Main> norepeat [1,1,2,5,5,6]
[1,2,5,6]
*Main> norepeat [1,1,2,5,5,6, 6]
[1,2,5,6]
*Main> norepeat [1,2,5,5,6, 6]
[1,2,5,6]
*Main> norepeat [1,2,5]
[1,2,5]
*Main>

\end{lstlisting}
\end{frame}


\begin{frame}[fragile]
\frametitle{Quick sort}
\begin{lstlisting}
qsort :: [Int] -> [Int]
qsort [] = []
qsort l = (qsort [x | x<-l, x < y]) ++ [y] ++ (qsort [x | x<-l, x>y]) where y = (head l)
\end{lstlisting}
\end{frame}

\begin{frame}[fragile]
\frametitle{Razdvajanje stringa u listu reči}
\begin{lstlisting}
findFirstWord :: [Char] -> ([Char], [Char])
findFirstWord "" = ("", "")
findFirstWord " " = ("", "")
findFirstWord s =
    let (remaining, others) = findFirstWord $ tail s
        x = head s
    in if x == ' ' then ("", tail s) else (x : remaining, others)

split :: [Char] -> [[Char]]
split "" = []
split s = let (x, y) = findFirstWord s
          in [x] ++ (split y)
\end{lstlisting}
\end{frame}

\begin{frame}[fragile]
\frametitle{Funkcija koja vraća dužinu stringa}
\begin{lstlisting}
strlen :: String -> Int
strlen s
    | s == "" = 0
        | otherwise = 1 + (strlen $ tail s)
\end{lstlisting}
\end{frame}


\frame{
\frametitle{Tipovi podataka}
\begin{itemize}
\item Definišu se tako što se navede ime tipa i njegovi konstruktori
\item data NoviTip = konstruktor1 | konstruktor2 | ... 
\item data Shape = Rectangle Int Int | Circle Int | Square Int 
\end{itemize}
}

\begin{frame}[fragile]
\frametitle{Primer funkcije nad tipom}
\begin{lstlisting}
data Shape = Circle Float | Rectangle Float Float 
area :: Shape -> Float 
area (Circle r) = r*r*3.14
area (Rectangle a b) = a*b
\end{lstlisting}
\end{frame}


\begin{frame}[fragile]
\frametitle{Primer - Zipper}
\begin{lstlisting}
data Zipper a = Zip [a] a [a] deriving (Show)

right :: Zipper a -> Zipper a
right (Zip l focus []) = Zip l focus [] 
right (Zip l focus r) = Zip (focus:l) (head r ) (tail r)


left :: Zipper a -> Zipper a
left (Zip [] focus r) = Zip [] focus r
left (Zip l focus r) = Zip (tail l) (head l) (focus:r)

\end{lstlisting}
\end{frame}


\begin{frame}[fragile]
\begin{lstlisting}
*Main> z = Zip [3, 2, 1] 4 [5, 6, 7]
*Main> right z 
Zip [4,3,2,1] 5 [6,7]
*Main> left $ right z 
Zip [3,2,1] 4 [5,6,7]
*Main> left $ left $ right z 
Zip [2,1] 3 [4,5,6,7]
*Main> left $ left $ left $ right z 
Zip [1] 2 [3,4,5,6,7]
*Main> left $ left $ left $ left $ right z 
Zip [] 1 [2,3,4,5,6,7]
*Main> left $ left $ left $ left $ left $ right z 
Zip [] 1 [2,3,4,5,6,7]
*Main> 
\end{lstlisting}
\end{frame}


\begin{frame}[fragile]
\frametitle{Binarno stablo}
\begin{lstlisting}
data TreeNode = Node {
        left :: Node,
        right :: Node,
        value :: Int
    } | EmptyNode

tolist :: TreeNode -> [Int]
tolist EmptyNode = []
tolist n = let l = tolist $ left n
               r = tolist $ right n
               v = value n
           in l ++ [v] ++ r

append :: TreeNode -> Int -> Node
append EmptyNode x = Node EmptyNode EmptyNode x
append n x = let v = value n
                 r = right n
                 l = left n
             in if x < v then Node (append l x) r v else Node l (append r x) v


fromlist :: [Int] -> TreeNode -> TreeNode
fromlist [] n = n
fromlist l n = let x = head l
               in fromlist (tail l) (append n x)

bstsort :: [Int] -> [Int]
bstsort l = tolist (fromlist l EmptyNode)
\end{lstlisting}
\end{frame}



\frame{
\frametitle{Monade}
\begin{itemize}
\item Način da se promeni stanje 
\item U realnom svetu, pojavljuje se potreba za moguč=ćnošću da se stanje može promeniti
\item Primer: IO
\item Monada uokviruje postojeći tip sa dodatnim informacijama koje se provlače kroz izračunavanje. 
\item Potrebno implementirati:
\begin{itemize}
\item Tip podataka 
\item return funkciju - a - > m a 
\item bind funkciju - m a -> (a -> m a) -> m a
\end{itemize}
\item return ne utiče na izračunavanje 
\item bind vezuje dva izračunavanja - poput kompozicije
\end{itemize}
}

\begin{frame}[fragile]
\frametitle{Primer * funkcije sa celim brojevima koje loguju operacije}
\begin{lstlisting}
data Debuggable = Debuggable (Int, [[Char]]) deriving (Show)


ret :: Int -> Debuggable 
ret x = Debuggable (x, [])

bind :: Debuggable -> (Int -> Debuggable) -> Debuggable
bind (Debuggable a b) f = Debuggable (y, z ++ b) where Debuggable (y,z) = f(a)

inc :: Int -> Debuggable  
inc x = Debuggable (x+1, ["Number increased by one"])
\end{lstlisting}
\end{frame}

\begin{frame}[fragile]
\begin{lstlisting}
*Main> ret 5 
Debuggable (5,[])
*Main> inc 5 
Debuggable (6,["Number increased by one"])
*Main> bind (ret 5) inc 
Debuggable (6,["Number increased by one"])
*Main> bind (bind (ret 5) inc) inc
Debuggable (7,["Number increased by one","Number increased by one"])
*Main> 
\end{lstlisting}
\end{frame}


\frame{
\begin{itemize}
\item Haskell ima sintaksni šećer za monade 
\item pomoću do-bloka je moguće pisati kod koji liči na kod u imperativnim jezicima 
\item Potrebno implementirati monade tako da ih Haskell prepozna kao takve 
\end{itemize}
}

\begin{frame}[fragile]
\frametitle{Implementacija monade tako da možemo koristiti Haskellov sintaksni šećer}
\begin{lstlisting}
import Control.Applicative -- Otherwise you can't do the Applicative instance.
import Control.Monad (liftM, ap)

data Debuggable a = Debuggable (a, [[Char]]) deriving (Show)
-- needed since newer versions of GHC
instance Functor Debuggable where
  fmap = liftM

instance Applicative Debuggable where
  pure  = return
  (<*>) = ap
------------------------

instance Monad Debuggable where 
  return x = Debuggable (x, [])
  (>>=) (Debuggable (a,b)) f = Debuggable (y, z ++ b) where Debuggable (y,z) = f(a)

inc :: Int -> Debuggable Int
inc x = Debuggable (x+1, ["Number increased by one"])
\end{lstlisting}
\end{frame}


\begin{frame}[fragile]
\frametitle{Primer upotrebe return i bind funkcija}
\begin{lstlisting}
*Main> (inc 5) >>= (\x -> (return x) >>= inc)
Debuggable (7,["Number increased by one","Number increased by one"])
*Main>
\end{lstlisting}
\end{frame}

\begin{frame}[fragile]
\frametitle{do-blokovi}
\begin{lstlisting}
inc2 :: Int -> Debuggable Int 
inc2 num = do 
  x <- inc num
  y <- inc x 
  return y 
\end{lstlisting}
\end{frame}

\begin{frame}[fragile]
\frametitle{Unit monada}
\begin{lstlisting}
import Control.Applicative -- Otherwise you can't do the Applicative instance.
import Control.Monad (liftM, ap)

data Unit a = Unit a deriving (Show)

instance Functor Unit where
  fmap = liftM

instance Applicative Unit where
  pure  = return
  (<*>) = ap

instance Monad Unit where 
  return x = Unit x
  (>>=) (Unit x) f = f x
\end{lstlisting}
\end{frame}

\begin{frame}[fragile]
\frametitle{Maybe monada}
\begin{lstlisting}
import Control.Applicative -- Otherwise you can't do the Applicative instance.
import Control.Monad (liftM, ap)

data Maybe a = Just a | Nothing

instance Functor Maybe where
  fmap = liftM

instance Applicative Maybe where
  pure  = return
  (<*>) = ap

instance Monad Maybe where 
  return x = Just x
  (>>=) (Nothing) _ = Nothing
  (>>=) (Just x) f = f x
\end{lstlisting}
\end{frame}

\begin{frame}[fragile]
\frametitle{IO monada - hello world}
\begin{lstlisting}
main :: IO ()
main = putStrLn "Hello world!" 

ghc hello.hs -o hello -dynamic 

\end{lstlisting}
\end{frame}


\begin{frame}[fragile]
\frametitle{Unos podataka preko stadardnog ulaza}
\begin{lstlisting}
main :: IO ()
main = do
  putStrLn "Enter your name"
  name <- getLine
  putStrLn $ "Your name is " ++ name
\end{lstlisting}
\end{frame}

\frame{
\frametitle{WTF?????}
\begin{itemize}
\item IO monada je hak 
\item IO moada je izlaz u realni svet 
\item IO monada čuva redni broj tako da funkcije se ne optimizuju od strane kompajlera već se uvek izvršavaju!
\end{itemize}
}


\begin{frame}[fragile]
\frametitle{Naš printStrLn} 
\begin{lstlisting}

{-# LANGUAGE ForeignFunctionInterface #-}

import Foreign.C.String
import Foreign.C.Types 

foreign import ccall safe "prototypes.h my_print_line"
    cMyPrintLine :: CString -> IO ()

myPrintLine :: String -> IO () 
myPrintLine str = do 
    c_str <- newCString str
    cMyPrintLine c_str



main = do 
    myPrintLine "hello!"


\end{lstlisting}
\end{frame}

\begin{frame}[fragile]
\frametitle{C kod}
\begin{lstlisting}

#include <stdio.h>
#include "functions.h"


void my_print_line(char* str) 
{
	printf("%s\n", str);
}


\end{lstlisting}
\end{frame}

\begin{frame}[fragile]
\frametitle{header}
\begin{lstlisting}
extern int my_rand();
extern void my_print_line(char*);
\end{lstlisting}
\end{frame}


\begin{frame}[fragile]
\frametitle{Implementacija rand funkcije}
\begin{lstlisting}

#include <stdlib.h>
#include <time.h>

int my_rand() 
{
	time_t t; 
	srand((unsigned) time(&t));
	return 1 + rand() % 5;
}

\end{lstlisting}
\end{frame}

\begin{frame}[fragile]
\begin{lstlisting}
foreign import ccall safe "prototypes.h my_rand"
    cMyRand :: IO CInt

myRand :: IO Int
myRand = do
    r <- cMyRand
    return ((fromInteger . toInteger) r)


guessToStr :: Int -> String
guessToStr 1 = "1"
guessToStr 2 = "2"
guessToStr 3 = "3"
guessToStr 4 = "4"
guessToStr 5 = "5"
guessToStr 6 = "6"


main = do
    myPrintLine "hello!"
    myPrintLine "hello!"
    r1 <- myRand
    r2 <- myRand
    myPrintLine $ guessToStr r1
    myPrintLine $ guessToStr r2

\end{lstlisting}
\end{frame}

\frame{
\frametitle{Zadaci}
\begin{itemize}
\item JSON parser 
\item Parsiranje izraza (5+2, (3+2)*5, 5*2 - 1, ...)
\item Web server sa REST API-em koji komunicira sa bazom
\end{itemize}
}
\end{document}
