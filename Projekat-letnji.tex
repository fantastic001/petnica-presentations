\documentclass{beamer} 

\usepackage[utf8]{inputenc}
\usepackage[T1]{fontenc}
\usepackage{graphicx}
\usepackage{multicol}

\usetheme{Berlin}
\begin{document}

\title{Istraživački projekti - HOW TO}
\author{Stefan Nožinić (stefan@lugons.org)}

\frame{
\titlepage
}

\frame{
	\frametitle{Uvod}
	\begin{itemize} 
		\item Cilj je učenje istraživačkog procesa kroz:
		\item Planiranje
		\item Implementaciju
		\item Evaluaciju
		\item Dokumentovanje 
		\item Prezentaciju rezultata
	\end{itemize}
}

\frame{
	\frametitle{Naučna metodologija}
	\begin{itemize}
		\item Postavljanje hipoteze - pretpostavke - zimski seminar
		\item Pravljenje plana implementacije i evaluacije - period posle zimskog a pre letnjeg
		\item Implementacija - letnji seminar
		\item Verifikacija implementacije - letnji seminar
		\item Prikupljanje rezultata - kraj letnjeg seminara
		\item Pisanje naučnog rada kao izveštaja o rezultatima istraživanja - poslednji dani letnjeg seminara i jesenji seminar
	\end{itemize}
}


\frame{
	\frametitle{Implementacija}
	\begin{itemize}
		\item U ovoj fazi mnogi projekti propadaju!
		\item Glavni razlozi: loše poznavanje teorije i/ili loše poznavanje programiranja
	\end{itemize}
}

\frame{
	\frametitle{Odabir alata}
	\begin{itemize}
		\item Uzmite nešto gde se snalazite 
		\item Jezik 
		\item Frejmvork
		\item Okruženje
		\item OS? Kompajler? Uređaj?
		\item Ako koristite novu tehnologiju:
		\begin{itemize}
			\item čitanje dokumentacije
			\item terminologija može da se ne poklapa sa terminologijom iz literature 
			\item "pesak" za novu tehnologiju pre projekta - da vam uđe "pod prste" 
		\end{itemize}
		\item KISS
		\item Koristiti proverene alate sa velikom zajednicom i dokumentacijom ako je moguće
	\end{itemize}
}

\frame{
	\frametitle{Planiranje implementacije}
	\begin{itemize}
		\item Podela na male module
		\item Jedan modul treba da radi jednu operaciju
		\item Smanjiti zavisnost između modula upotrebom "interfejsa"
		\item Linearna proširivost sistema
		\item Napravite TO-DO listu gde svaki zadatak traje najviše 2 sata
	\end{itemize}
}

\frame{
	\frametitle{Dodatni alati i metodologija}
	\begin{itemize}
		\item unit testovi 
		\item version control system (git, svn, mercurial, ...)
		\item debugger (gdb, ...)
		\item profiler (valgrind, plop, ...)
		\item build system (autotools, cmake, ...)
	\end{itemize}
}

\frame{
	\frametitle{Evaluacija}
	\begin{itemize}
		\item Radi se za svaku metriku za svaki algoritam
		\item Podatke čuvati tabelarno
		\item Grafikoni (bar plot, kriva, histogram, pita, mapa inteziteta, graf, ...)
		\item Primeri koda u novom jeziku
		\item Animacija simulacije
	\end{itemize}
}

\frame{
	\frametitle{Pisanje rada}
	\begin{itemize}
		\item Ključne reči
		\item Naslov 
		\item Uvod
		\item Opis metoda i algoritama
		\item Prezentacija rezultata
		\item Diskusija, Zaključak i predlozi za unapređenje
		\item Apstrakt
	\end{itemize}

}

\frame{
	\frametitle{Za kraj}
	\begin{itemize}
		\item Cilj je učenje kroz projektni rad 
		\item Konferencija nije cilj već alat 
		\item Pitanja?
	\end{itemize}
}
\end{document}
